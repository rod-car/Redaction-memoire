% \begin{itemize}
%     \item Indexer des documents (PDF, WORD, Texte sur des images) (50 \%)
%     \item Extraction automatique de titre, date, auteur, \dots
%     \item S'authentifier (Terminé)
%     \item Créer un compte (Terminé)
%     \item Télécharger des documents
%     \item Visualiser des documents
%     \item Créer des bibliographie automatique
%     \item Déposer un documents (Thèse pour un étudiant) (Terminé)
%     \item Déposer des documents multiples (Administrateur seulement)
%     \item Supprimer un document  (Des documents)
%     \item Protéger un document
%     \item Rechercher par mots clés (Terminé)
%     \item Recherche d'un image
%     \item Documents payant
%     \item Accès enseignant
%     \item Accès étudiant (Dashboard uniquement terminé)
%     \item Catégorisation automatique d'un document déposé
%     \item Filtre par année
%     \item Filtre par catégorie
%     \item Recherche par titre, contenu, résumé, auteur (Terminé sauf pour le resumé)
%     \item Pagination de résultat (En cours)
%     \item Mise en place d'espace utilisateur
%     \item Définition de seuil de résultat par défaut (Terminé)
%     \item Définition de seul de résultat par rapport au préférence de l'utilisateur
%     \item Filtre de requête en utilisant des caractères spécifique
%     \item Filtre par type d'université (Publique, privé), université (Tana, Toa, Tol, \dots)
%     \item Suggestion de mots clés
%     \item Suggestion des travaux connexes
%     \item Possibilité d'avoir un aperçu du document
%     \item Système orienté web (Terminé)
% \end{itemize}


% \begin{algorithm}
%     \caption{Faire une recherche}
%     \begin{algorithmic}[1]
%         \Procedure{index}{$request: HttpRequest$} \Comment{Page de recherche et traite l'appariément requête-documents}
%             \State
%             \State \textbf{Arguments:}
%             \State $request$: La requête contenant les mots clés de l'utilisateur
%             \State
%             \State \textbf{Renvoie:}
%             \State $HttpResponse$: La page de recherche ou la page de résultats de recherche
%             \State
%             \If{("query" dans $request.GET$)}
%                 \State $query \gets request.GET['query']$
%             \Else
%                 \State $query \gets ""$
%             \EndIf
%             \State
%             \If{("filter" dans $request.GET$)}
%                 \State $filter\_field \gets request.GET['filter']$
%             \Else
%                 \State $filter\_field \gets "content"$
%             \EndIf
%             \State
%             \If{($query \neq ""$)}
%                 \State $start \gets maintenant()$
%                 \State $end \gets None$
%                 \State $documents \gets []$
%                 \State
%                 \State $inverted\_indexes \gets Index.objects.all()$
%                 \State
%                 \If{($inverted\_indexes.count() > 0$)}
%                     \State $terms \gets [inverted\_index.term$ \textbf{pour} $inverted\_index$ \textbf{dans} $inverted\_indexes]$
%                     \State $terms \gets [preprocess\_text(term)$ \textbf{pour} $term$ \textbf{dans} $terms]$
%                     \State
%                     \State $vectorizer \gets TfidfVectorizer()$
%                     \State $vectorizer.fit\_transform(terms)$
%                     \State
%                     \State $correct\_query\_terms \gets []$
%                     \State $query\_terms \gets tokenize(query)$
%                     \State
%                     \For{$term$ \textbf{dans} $query\_terms$}
%                         \State $correct\_term \gets correct\_word(term)$
%                         \If{$correct\_term$ est \textbf{None}}
%                             \State $correct\_query\_terms$.append($term$)
%                         \Else
%                             \State $correct\_query\_terms$.append($correct\_term$)
%                         \EndIf
%                     \EndFor
%                     \State
%                     \State $query\_terms \gets correct\_query\_terms$
%                     \State $query\_vector \gets vectorizer.transform([" ".join(query\_terms)]).toarray()[0]$
%                     \State
%                     \If{($filter\_field == "content"$)}
%                         \State $selected\_docs \gets search(query\_terms=query\_terms, query\_vector=query\_vector)$
%                         \If{($len(selected\_docs) > 0$)}
%                             \State $documents \gets Documents.findByIdsWithScore(ids = selected\_docs)$
%                         \EndIf
%                     \EndIf
%                     \State
%                     \If{($filter\_field == "title"$)}
%                         \State $selected\_docs \gets searchBy(query\_terms=query\_terms, query\_vector=query\_vector, field="title")$
%                         \If{($len(selected\_docs) > 0$)}
%                             \State $documents \gets Documents.findByIdsWithScore(ids = selected\_docs)$
%                         \EndIf
%                     \EndIf
%                     \State
%                     \If{($filter\_field == "author"$)}
%                         \State $query\_terms \gets query.split()$
%                         \State $selected\_docs \gets searchBy(query\_terms=query\_terms, query\_vector=query\_vector, field="author")$
%                         \If{($len(selected\_docs) > 0$)}
%                             \State $documents \gets Documents.findByIdsWithScore(ids = selected\_docs)$
%                         \EndIf
%                     \EndIf
%                 \Else
%                     \State \textbf{pass}
%                 \EndIf
%                 \State
%                 \If{('page' dans $request.GET$)}
%                     \State $page\_number \gets request.GET['page']$
%                 \Else
%                     \State $page\_number \gets 1$
%                 \EndIf
%                 \State
%                 \If{('perpage' dans $request.GET$)}
%                     \State $per\_page \gets request.GET['perpage']$
%                 \Else
%                     \State $per\_page \gets 5$
%                 \EndIf
%                 \State
%                 \State $paginator \gets Paginator(object\_list=documents, per\_page=per\_page)$
%                 \State $page\_obj \gets paginator.get\_page(number=page\_number)$
%                 \State
%                 \State $end \gets maintenant()$
%                 \State $response\_time \gets (end - start).total\_seconds()$
%                 \State
%                 \State \textbf{Renvoyer} $render(request=request, template\_name="client/results.html", context=\{... \})$
%             \Else
%                 \If{($request.session.has\_key('query')$)}
%                     \State $request.session.pop('query')$
%                 \EndIf
%                 \State
%                 \State \textbf{Renvoyer} $render(request=request, template\_name="client/index.html", context=\{... \})$
%             \EndIf
%         \EndProcedure
%     \end{algorithmic}
% \end{algorithm}

% \begin{algorithm}
%     \caption{Mise à jour des Vecteurs TF-IDF}
%     \begin{algorithmic}[1]
%     \State \textbf{Données :} Indexes Inversés, Valeurs TF-IDF, Documents
%     \State \textbf{Résultat :} Mise à jour des Vecteurs TF-IDF
%     
%     \State \textbf{Récupérer tous les index inversés :}
%     \State inverted\_indexes $\leftarrow$ Index.objects.all()
%     \State terms $\leftarrow$ [inverted\_index.term \textbf{pour} inverted\_index \textbf{dans} inverted\_indexes]
%     \State terms $\leftarrow$ [preprocess\_text(term) \textbf{pour} term \textbf{dans} terms]
%     
%     \State \textbf{Initialiser un vecteuriseur TF-IDF :}
%     \State vectorizer $\leftarrow$ TfidfVectorizer()
%     \State vectorizer.fit\_transform(terms)
%     
%     \State \textbf{Récupérer toutes les valeurs TF-IDF de la base de données :}
%     \State tfidf\_values $\leftarrow$ TfidfValues.objects.all()
%     \State doc\_ids $\leftarrow$ [tfidf\_value.document\_id\_id \textbf{pour} tfidf\_value \textbf{dans} tfidf\_values]
%     
%     \For{chaque doc\_id \textbf{dans} doc\_ids}
%         \State document\_terms $\leftarrow$ Documents.objects.get(pk=doc\_id).content.split()
%         \State document\_vector $\leftarrow$ vectorizer.transform([" ".join(document\_terms)]).toarray()[0]
%         
%         \State document\_vector $\leftarrow$ np.round(document\_vector, 4)
%         \State document\_vector\_json $\leftarrow$ json.dumps(document\_vector.tolist()) \textbf{  // Convertir en JSON}
%         
%         \State tfidf\_value $\leftarrow$ TfidfValues.objects.get(document\_id=doc\_id)
%         \State tfidf\_value.tfidf\_vectors $\leftarrow$ document\_vector\_json
%         \State tfidf\_value.save()
%     \EndFor
%     \end{algorithmic}
% \end{algorithm}


% \begin{algorithm}
%     \caption{Gérer le Fichier Téléchargé}
%     \begin{algorithmic}[1]
%     \Function{handle\_uploaded\_file}{$f, dossier$}
%         \State \textbf{Ouvrir} $destination$ \textbf{en mode} "wb+" \textbf{comme} $destination$
%         \For{\textbf{chaque} $chunk$ \textbf{dans} $f.chunks()$}
%             \State $destination.write(chunk)$
%         \EndFor
%     \EndFunction
%     \end{algorithmic}
% \end{algorithm}