\chapter{Simulation et Réalisation}
\section{I-Karoka}
% Expliquer les principes dondamentale du SRI a mettre en place (Introduction)
\subsection{Liste des fonctionnalités}
\begin{itemize}
    \item Indexer des documents (PDF, WORD, Texte sur des images)
    \item S'authentifier
    \item Créer un compte
    \item Télécharger des documents
    \item Visualiser des documents
    \item Créer des bibliographie automatique
    \item Déposer un documents (Thèse pour un étudiant)
    \item Déposer des documents multiples (Administrateur seulement)
    \item Supprimer un document  (Des documents)
    \item Protéger un document
    \item Rechercher par mots clés
    \item Recherche d'un image
    \item Documents payant
    \item Accès enseignant
    \item Accès étudiant
    \item Catégorisation automatique d'un document déposé
    \item Filtre par année
    \item Filtre par catégorie
    \item Recherche par titre, contenu, résumé, auteur
    \item Pagination de résultat
    \item Mise en place d'espace utilisateur
    \item Définition de seuil de résultat par défaut
    \item Définition de seul de résultat par rapport au préférence de l'utilisateur
    \item Filtre de requête en utilisant des caractères spécifique
    \item Filtre par type d'université (Publique, privé), université (Tana, Toa, Tol, \dots)
    \item Suggestion de mots clés
    \item Suggestion des travaux connexes
    \item Possibilité d'avoir un aperçu du document
    \item Système orienté web
\end{itemize}

\section{Approche utilisée}
\subsection{UML}
\lipsum[1-2]

\section{Technologies utilisé}
\subsection{Python}
\subsection{Librairie Spacy, NLTK}
\subsection{Librairie ScikitLearn}
\subsection{Framework Django}


\subsection{MySQL}
\lipsum[1-2]

\section{Outils utilisée}
\lipsum[1-2]

\subsection{Visual Studio Code}
\lipsum[1-2]