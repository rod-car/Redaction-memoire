\chapter*{Introduction générale}
\addcontentsline{toc}{chapter}{Introduction générale}
\markboth{INTRODUCTION GÉNÉRALE}{}

\section*{Vue d'ensemble}
\addcontentsline{toc}{section}{Vue d'ensemble}
La recherche d'information (\emph{RI}) est le fait de rechercher une information dans une base de documents (corpus) a travers un système de recherche d'information ou moteur de recherche. C'est aussi trouver un document pertinent qui satisfait un besoin d'information de l'utilisateur. Dans le cadre de ce devoir, on vise a satisfaire le besoin d'information qui est de trouver des travaux de recherche connexe (articles, thèses doctorat, mémoire de master, etc.) a un thème de recherche afin de pouvoir rédiger la cadre théorique ou état de l'art. Il est donc nécessaire de consulter des ressources sur internet par le biais des moteurs de recherches ou des ressources physiques (Documents imprimés).

Un Système de Recherche d'Information (\emph{SRI}) ou moteur de recherche est un ensemble de programmes qui stock les documents et permet de donner des résultats suivant un degré de pertinence en fonction d'une requête de l'utilisateur.

Le modèle vectoriel est un modèle mathématique utilisé dans la recherche d'information pour définir la comportement de la Système de Recherche d'Information et définit l'algorithme de similarité qui détermine la pertinence des documents dans la collection par rapport a la requête de l'utilisateur afin de classer les résultats (liste des documents) retournés a l'utilisateur.

\section*{Contexte de la recherche de thèses malagasy}
\addcontentsline{toc}{section}{Contexte de la recherche de thèses malagasy}
Pour rechercher des thèses malagasy (soutenue dans les universités a Madagascar), les moteurs de recherche académique comme Google scholar, HAL, Theses.fr, etc., ne sont pas véritablement efficace dans le sens que le nombre des documents (Thèses, Articles) malagasy qui y sont indexé n'est pas suffisant. Cette manque peut être dû au fait que ces ressources ne sont pas déposé en ligne pour faciliter l'accès en ligne, alors qu'ils sont difficile a indexer par les moteurs de recherche. Ce qui implique qu'il faudra passer par un SRI local pour héberger ces documents ainsi que de les rechercher.

Actuellement, le premier système en place est \og~Thèse malgache en ligne~\fg{} qui représente \textbf{31 160} documents provenant des six universités publiques dont \textbf{197} pour l'Université d'Antsiranana, \textbf{1 190} pour l'Université de Mahajanga, \textbf{973} pour l'Université de Toamasina, \textbf{27 789} pour l'Université d'Antananarivo, \textbf{338} pour l'Université de Fianarantsoa et \textbf{673} pour l'Université de Toliara \citep{these-malgache-en-ligne}.

Ces documents sont pas encore suffisant, vu le nombre d’étudiants qui termine leurs études en master, doctorat, etc. D'autre part, l'université d'Antananarivo occupe environ le \emph{89\%} du nombre des documents totale et le \emph{11\%} pourcent restant pour les autres universités. D'autre source tel que des bibliothèques en ligne, et hors-lignes (documents papiers, numériques mais qui ne sont pas accessible en ligne) des universités pouvant être utilisés, mais la plupart de ces systèmes (version en ligne) sont sans interface pour faire des recherches par mots clés. Ces lacunes sont la source de motivation de ce mémoire pour apporter une amélioration.

\section*{Problématique}
\addcontentsline{toc}{section}{Problématique}
Alors l'identification et l'exploitation des thèses et mémoires malagasy dans un domaine spécifique d'un sujet de recherche sont encore un défi en raison de limitations des ces systèmes de recherche existants. Certains de ces systèmes sont souvent peu performants, difficiles à utiliser en raison d'une interface utilisateur peu conviviale, et limités en terme de ressources et limités aux ressources des six universités publiques malgaches (Si nécessaire). Par conséquent, il est difficile de mener une recherche approfondie et d'exploiter pleinement les résultats disponibles. De plus, ces travaux de recherche sont souvent peu visibles sur les systèmes de recherche d'information populaires tels que \emph{Google Scholar}, \emph{Theses.fr}, \emph{HAL}, \emph{Mémoire online}, etc. Cette situation limite l'exploitation et la valorisation des travaux pertinents réalisés par les étudiants et chercheurs malagasy. Et enfin, la sécurité de ces fruits de recherche est un véritable défi pour les SRI comme la protection de droit d'auteur, protection contre le vol de compétence, la sécurité de l'hébergement, etc.

\section*{Objectifs de la mémoire}
\addcontentsline{toc}{section}{Objectifs de la mémoire}
L'objectif principal est de permettre aux chercheurs malagasy d'accéder facilement aux travaux de recherche pertinents et de promouvoir la recherche locale avec une meilleur sécurité. Ce mémoire vise donc a développer un système de recherche d'information (\emph{SRI}), basé sur le modèle vectoriel, permettant de résoudre les problèmes rencontrés dans l'accès aux travaux de recherche malagasy tels que les mémoires et thèses. Ce système doit être convivial, performant et regrouper tous les documents provenant des établissements d'enseignement supérieur publics et privés. Ainsi ce système propose un accès plus sécurisé pour les documents pour éviter le vol et protéger le droits d'auteur par une mise en place de système d'authentification, un système d'accès par une organisation ou université, et un système de paiement pour un document payant. Pour minimiser les risques, ce SRI doit être hébergé a Madagascar pour protéger ces ressources. 

\section*{Organisation du devoir}
\addcontentsline{toc}{section}{Organisation du devoir}
Pour mieux élaborer ce thème, ce devoir va se dérouler en quatre partie. La première partie concerne l'état de l'art qui est divisé en quatre chapitre tel que la Recherche d'Information où on va voir les bases et les principes de la recherche d'information, ses facteurs d'influences, ainsi que ses différents modèles; le Système de Recherche d'information (SRI) où on analysera le fonctionnement d'un moteur de recherche, ses défis, ainsi que don évaluation; le Traitement de Langage Naturel (TLN) où on va analyser les méthodes de traitement de langage naturel et ses utilités vis-a-vis de la recherche d'information; et enfin le Modèle Vectoriel (Vector Space Model) qui est la méthode (modèle) choisi dans le cadre de ce mémoire où on va analyser ses principes, ses avantages ainsi que ses limites. La deuxième partie on va détailler la simulation et la réalisation du SRI proposé ci-dessus, ainsi que la liste de ces fonctionnalités et les outils de développement utilisé, et on analysera ses performances et ces limites. La dernière partie qui est la conclusion générale ou on va conclure notre travail de recherche ainsi de donner des perspectives.