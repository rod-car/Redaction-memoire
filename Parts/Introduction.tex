\chapter*{Introduction générale}
\addcontentsline{toc}{chapter}{Introduction générale}
\markboth{INTRODUCTION GÉNÉRALE}{}

\section*{Vue d'ensemble}
\addcontentsline{toc}{section}{Vue d'ensemble}
La recherche d'information (RI) c'est le fait de rechercher une information dans une base de documents (Corpus) a travers un système de recherche d'information ou moteur de recherche. C'est aussi trouver un document pertinent qui satisfait un besoin d'information de l'utilisateur.

Un Système de Recherche d'Information (SRI) ou moteur de recherche est un ensemble de programmes qui stock les documents et permet de donner des résultats suivant un degré de pertinence en fonction d'une requête de l'utilisateur.

Le modèle vectoriel est un modèle mathématique utilisé dans la recherche d'information pour définir la comportement de la Système de Recherche d'Information et définit l'algorithme de similarité entre la requête de l'utilisateur et les documents ainsi que le classement des documents.

Pour qu'un étudiant puissent rédiger et travailler sur son travail de recherche, il est important de trouver les documents concernant le thème en question (Articles, Thèses, Mémoires, etc.) pour rédiger l'état de l'art ou cadre théorique. Il est donc nécessaire de consulter des ressources sur internet par le biais des moteurs de recherches ou ressources physiques (Documents imprimés).

\section*{Contexte de la recherche de thèses malagasy}
\addcontentsline{toc}{section}{Contexte de la recherche de thèses malagasy}
Pour rechercher les thèses malagasy, les moteurs de recherche académique, géant comme Google et Google scholar n'est pas suffisamment efficace dans le sens qu'il y a pas beaucoup des base de documents (Thèses, Articles) malagasy en ligne et que ces moteurs n'arrivent pas a indexer ces documents dans leur base. Il faudra passer donc par un SRI local pour héberger ces documents ainsi que de les rechercher.

Actuellement, le premier système en place c'est \og Thèse malgache en ligne \fg{} qui représente actuellement (Stock) 31160 documents provenant des six universités publiques dont 973 pour l'Université de Toamasina, 27789 pour l'Université d'Antananarivo, 673 pour l'Université de Toliara, 1190 pour l'Université de Mahajanga, 338 pour l'Université de Fianarantsoa et 197 pour l'Université d'Antsiranana \citep{these-malgache-en-ligne}.

Ces documents sont pas encore suffisant, vu le nombre d’étudiants qui obtient leur diplôme de Master. Ainsi que l'université d'Antananarivo occupe environ de 89\% du nombre des documents totale et les autres le 11\% pourcent restant. Il y a aussi des bibliothèques en ligne des universités mais la plupart sans interface pour faire des recherches par mots clés. Ces lacunes sont la source de motivation de ce mémoire pour apporter une amélioration.

\section*{Problématique}
\addcontentsline{toc}{section}{Problématique}
Alors l'identification et l'exploitation des thèses et mémoires malagasy dans un domaine spécifique d'un sujet de recherche sont encore un défi en raison de limitations des ces systèmes de recherche existants. Certains de ces systèmes sont souvent peu performants, difficiles à utiliser en raison d'une interface utilisateur peu conviviale, et limités en terme de ressources et limités aux ressources des six universités publiques malgaches (Si nécessaire). Par conséquent, il est difficile de mener une recherche approfondie et d'exploiter pleinement les résultats disponibles. De plus, ces travaux de recherche sont souvent peu visibles sur les systèmes de recherche d'information populaires tels que Google Scholar ou Mémoire Online. Cette situation limite l'exploitation et la valorisation des travaux pertinents réalisés par les étudiants et chercheurs malagasy. Et puis la sécurité de ces fruits de recherche est un véritable défi pour les SRI (Moteur de recherche) comme protection de droit d'auteurs, vols de compétence, quel serveur doit héberger ces documents pour éviter les vols de ces fruits de recherche, \dots

\section*{Objectifs du mémoire}
\addcontentsline{toc}{section}{Objectifs du mémoire}
L'objectif de ce mémoire est de développer un système de recherche d'information, sous la forme d'un moteur de recherche, basé sur le modèle vectoriel, qui résout les problèmes rencontrés dans l'accès aux travaux de recherche malagasy tels que les mémoires et thèses. Ce système doit être convivial, performant et regrouper tous les documents provenant des établissements d'enseignement supérieur publics et privés. Ainsi ce système propose un accès plus sécurisé pour les documents pour éviter le vol et protéger le droits d'auteur par une mise en place de système d'authentification, et une système de paiement pour un document payant. Ce SRI doit être hébergé a Madagascar pour mieux protéger ces ressources. L'objectif principal est de permettre aux chercheurs malagasy d'accéder facilement aux travaux de recherche pertinents et de promouvoir la recherche locale avec une meilleur sécurité.

\section*{Organisation du devoir}
\addcontentsline{toc}{section}{Organisation du devoir}
Pour mieux élaborer ce thème, ce devoir va se dérouler en quatre partie. La première partie c'est l'état de l'art où on va analyser ou on en est actuellement sur la Recherche d'Information et la Système de Recherche d'Information ainsi que ses modèles. On va aussi voir la Traitement de Langage Naturel (Natural Langage Processing ou NLP). La seconde partie concerne la méthodologie où on va analyser et détailler la méthode vectoriel (Modèle Vectoriel ou Vector Space Model), ainsi que l'approche sémantique. Dans la quatrième partie on va voir l'étape de la réalisation de ce Système de Recherche d'Information ainsi que ces fonctionnalités et détails. Et la quatrième et dernière partie la conclusion et perspective. 